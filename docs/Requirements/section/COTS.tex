\section{COTS | Components and Technical Constraints}
\label{sec:COTS}
%TODO: Check 4.5
Android is a Linux-based operating system for mobile devices such as
smartphones and tablets. The Android operating system is an open source project released under the Apache License. Android's user interface is based on direct manipulation using touch inputs that loosely correspond to real-world actions. The application layer framework include
Java-compatible libraries, which allows developers to program Android
applications in Java, using the Android SDK\@. Therefore, we are bound to the
constraints that Java apply in general, as well as the Android Framework and
the hardware limitations of the physical devices. 

\subsection{Touch Screen}
Although some Android devices have physical keyboards, most of them uses a
touch screen as their main input device. It is therefore important that any
interaction with the game must be designed and arranged in a manner that
supports the screen constraints. 

The controls of our game is based on touch input from the player. It is therefore important that the buttons and other control objects are of a suitable size (not too small) and that all text is written in a font which is easy to read.

\subsection{Screen Resolution}
The screen resolution on Android devices vary greatly. To deal with this, we
will add images of different resolution. The Android compiler will
automatically compile images with the correct resolution, as long as they are
provided and correctly placed within the source code.

We will have to deal with this through programmatically changing the board size and position by checking the screen size of the device. 

\subsection{Android device only}
The game will only run on Android mobile devices and Android emulators.

\subsection{Memory Constraints}
Mobile devices generally have a limited amount of memory. This will most likely
not be an issue, but still something we need to be aware of.

We don't see any issues with this, since the game is very lightweight.

\subsection{Server performance}
Our server will be a lightweight server with limited resources. The amount of
concurrent connections the server may handle will be something we must take
notice of.
