\section{Architectural tactics}
\label{sec:architecturaltactics}

\subsection{Modifiability} 
We plan to make modifications to our game, either during the project lifecycle or after the project is finished. Therefore it is important to focus on modifiability from the beginning and maintain this focus throughout the development lifecycle. 
We have chosen the MVC pattern, since it separates logic and interface in a useful manner. This separation will give higher modifiability.

\subsection{Testability}
Separating the user interface from the logic will make it easier to test components during development. We will also separate the server from the clients in order to test scalability and server performance under expected and heavy load.


\subsection{Availability} 
The game logic runs locally on each device, but relies on Internet connection to join games with other players. The multiplayer experience is achieved by setting up the server as a RESTful web service, providing the client with its required data. The client will be able to ask the server for available games, and return a query for the desired game. Each player will get its own shuffeled board when entering a game. The server will also keep all the players updated of the game state, and will send a notification if someone has won.


\subsection{Performance}
Our system will consist of a remote server and local clients. When measuring performance we will have to determine the system's reliability when involving heavy loading and time-cirtical response requirement, since the communication between players will be handled off the clients. 

Data transfer has to be set to a minimum to reduce the impact of delay between clients and server. The clients itself wouldn't have much load while its operative. Our challenge will therefore be on the server side. Mostly by handling large amount of users and network loads. 


%TODO: Write something about performance on server


\subsection{Usability} 
We will make our best efforts in order to follow the guidelines for Android user interfaces, in order to make the game intuitive for users. Since our target audience are experienced smartphone users, it will be necessary to have a nice and understandable user interface. 