\section{Architectural tactics}
\label{sec:architecturaltactics}
%TODO: Do this part better

\subsection{Modifiability} 
Modifiability is important in order to ensure the ability of changing the system during or after it's development. A system with low modifiability will often result in unhappy user or systems being replaced in total. 

We plan to make modifications to our game either during the project lifecycle, or after the project is finished. It is therefore important to focus on modifiability from the beginning, and maintain this focus throughout the development lifecycle. We have chosen the Model-View-Controller pattern, since it separates logic and the user interface in a useful manner. This separation
will give higher modifiability.

\subsection{Testability}
Testability is important in order to test that the system works as expected. Not only should the system in total work as expected, each smaller component of the system should work as expected. By making a system with high testability it will be easier to find and repair errors when they occur.

Separating the user interface from the logic will make it easier to test components during development. We will also separate the server from the client in order to test scalability and server performance under expected and heavy load.

\subsection{Availability} 
In this day and age, it is expected that all systems have a high level of availability and a low response time. Consumers want the system to be there whenever they want, both day or night. 

The game logic runs locally on each device, but relies on an internet connection to join games with other players. The multiplayer experience is achieved by setting up the server as a RESTful web service, providing the client with its required data. Each player will get their own shuffled board when entering a game. The server will also keep all the players updated about the game state by sending a notification whenever someone wins the game.

\subsection{Performance}
Our system will consist of a remote server together with local clients and the communication between players will be handled by the server. Because of this, when measuring performance, we will have to determine the system's reliability during heavy loads and time-critical response requirements.

Data transfer has to be set to a minimum to reduce the impact of delay between
the clients and the server. The client itself won't have much load while it's
operative, although we do have to keep in mind that the client needs an
active connection to the server while playing. Most of the challenge will
therefore be server side, mostly by handling large amount of users and network
loads.

\subsection{Usability} 
In the modern age of 2013 an application is lost if the user is not able to learn the application quickly and operate it without any major difficulties. A high level of usability is expected of all systems. In order to help developers there are design standards for Android applications\cite{website:android}.

We will make our best efforts in order to follow the guidelines for Android user interfaces design, in order to make the game intuitive for users. Since our target audience are experienced smartphone users, it will be necessary to have a nice and understandable user interface.
