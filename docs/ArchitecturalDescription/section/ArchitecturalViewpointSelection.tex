\section{Architectural viewpoint selection}
\label{architecturalviewpointselection}
%Finished
For this project we have chosen to use two primary architectural viewpoints and
one secondary. The main viewpoints will be development viewpoint and the logic
viewpoint. The secondary one will be the process viewpoint.

We initially elected to remove the physical viewpoint. Even though our game
will run on different physical devices (there is at least $3,997$ distinct
device models running Android\cite{website:androidfrag}), all of the
interaction with the hardware is abstracted by the JVM and the Android
operating system. Since we are going to use a server/client model for network
communication, however, we ended up with a simple physical view showing how
the client and server are connected.

\subsection{Logic View}
\begin{description}
	\item[Purpose:]{The logic view is concerned with the functionality that the
    system provides to end-users.}
	\item[Stakeholders:]{Players, Developers, Course Staff, Abakus and
    Online.}
	\item[Form of description:]{Class diagram using 4+1 model description.}
\end{description}

\subsection{Development View}
\begin{description}
	\item[Purpose:]{The development view illustrates a system from a programmers
    perspective, and is concerned with software management.}
	\item[Stakeholders:]{Developers, Course Staff.}
	\item[Form of description:]{Architecture layout diagram.}
\end{description}

\subsection{Process View}
\begin{description}
	\item[Purpose:]{The process view deals with the dynamic aspect of the system,
    explains the system processes and how they communicate and focuses on the
    runtime behaviour of the system.}
	\item[Stakeholders:]{Players, Developers, Course Staff, Abakus and Online.}
	\item[Form of description:]{Process blueprint with UML 2.0 notation.}
\end{description}

\subsection{Physical View}
\begin{description}
  \item[Purpose:]{The physical view describes the physical deployment of the
    system.}
  \item[Stakeholders:]{Developers, Course Staff and System Administrators (for
    instance Abakus and Online)}
  \item[Form of description:]{Diagram showing how the server and client are
    connected, and how functionality are divided between them, describes using UML-like notation.}
\end{description}
