\section{Architectural Drivers}
\label{sec:architecturaldrivers}
%Finished

\subsection{Short implementation time}
We only have 6 weeks, including the easter holiday, to finish implementing the
game to a playable state. This means that the basic gameplay needs to be easy
to implement. When the basics are in place we can choose which parts, if any,
we want to expand and/or modify.

\subsection{Developer experience}
All of the developers in the group have experience programming; some more than
others. None of the group members have any prior experience in game
development, however, except for the Pong game we implemented in the
introduction to the course. We will thus have to learn some new things during
the course of this project. For this reason, we want the basic gameplay of our
game to be simple, while more advanced features are a bonus. Extensibility and
modifiability are therefore both very important for this project.

\subsection{Release to Google Play}
We have a goal of releasing the game to Google Play. The architecture must thus
be modifiable and extendable, in order to accommodate bugfixes and updates.

\subsection{Target Audience}
The target audience for the game are students at NTNU\@. To begin with we hope
to have Abakus and Online, the two student societies for computer science
students at NTNU, use this game at their company presentations. While this is a
fairly narrow target audience, we hope to expand this audience in later updates.
The target audience has extensive knowledge of smartphones and many are great
programmers. It will thus be important for the game to follow Android user
interface guidelines and feel intuitive and native to use. Usability and
availability is very important.

\subsection{The Android Framework}
It is fair to assume that the architecture will be affected by constraints
imposed by the underlying Android framework that we will use.
