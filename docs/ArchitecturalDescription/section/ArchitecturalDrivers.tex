\section{Architectural Drivers} 
\label{sec:architecturaldrivers}


\subsection{Short implementation time}
We only have 6 weeks, counting the easter holiday, to implement the game to a playable application. This means that the basic gameplay need to be easy to implement. When the basics are in place, we can choose which part(s) we want to expand/modify.

\subsection{Developer experience}
All of the developers have experience programming, some more than others, however, none of the developers have any prior experience developing games (except for Pong). We will have to learn some new things along the way. Thus we want the basic gameplay to be simple, while more advanced features are a bonus. Therefore extendibility and modifiability are both very important for this project.

\subsection{Release to Google Play}
We have a goal of releasing the game to Google Play. Therefore the architecture needs to be modifiable and extendable in order to accommodate bugfixes and updates. 

\subsection{Target Audience}
The target audience for the game are students at NTNU. In the start we hope to make Online Linjeforening and Abakus Linjeforening use this game at their presentations. While this is a very narrow target audience, we hope to expand this audience in later updates. The target audience have extensive knowledge of smartphones, and many have great skills in programming, therefore it will be important that the game follows Android user interface guidelines and feels intuitive and native to use. Usability and availability is very important.

\subsection{The Android Framework}
It is fair to assume that the architecture will be affected by constraints imposed by the underlying Android framework that we will use.
